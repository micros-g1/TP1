\documentclass[10pt,a4paper]{article}
\usepackage[utf8]{inputenc}
\usepackage[spanish]{babel}
\usepackage{amsmath}
\usepackage{amsfonts}
\usepackage{amssymb}
\usepackage{graphicx}
\usepackage[left=2cm,right=2cm,top=2cm,bottom=2cm]{geometry}
\author{Lisandro Alvarez}
\begin{document}


\section{Perifericos}
\subsection{Encoder Rotativo}

\subsection{Lector}
Read ID intenta leer el codigo de la tarjeta. 
\subsection{Display}
Vamos a estructurar el codigo en dos capas: \emph{capa de bajo nivel} e \emph{interfaz}.
\subsubsection{Interfaz}
Aca vamos a definir y presentar las funciones publicas que puede usar el usuario:
\begin{itemize}
\item \textbf{Marquesina: }Recibe un string, la dirección de shifteo y la velocidad de shift. 
\item Clear all
\item Escribir
\item Shift
\end{itemize}
\subsection{Puerta}
Solo dos funciones, bloquear y desbloquear puerta. Al desbloquear se debera indicar el tiempo que permanecerá desbloqueada.
\section{Varios}
\subsection{Interrupciones SysTick}
SysTick nos da la posibilidad de linkear solo UN callback. Este callback se encarga de aumentar el contador de ticks, y de llamar a los handlers asociados a interrupciones de SysTick. Estas funciones handlers estaran almacenadas en un arreglo de punteros a funcion. Entonces este unico callback reocrre el arreglo(ver de que dimensiones) y llamar a los handlers linkeados.
\end{document}